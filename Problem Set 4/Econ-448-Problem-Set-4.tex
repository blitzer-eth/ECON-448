% Options for packages loaded elsewhere
\PassOptionsToPackage{unicode}{hyperref}
\PassOptionsToPackage{hyphens}{url}
\documentclass[
]{article}
\usepackage{xcolor}
\usepackage[margin=1in]{geometry}
\usepackage{amsmath,amssymb}
\setcounter{secnumdepth}{-\maxdimen} % remove section numbering
\usepackage{iftex}
\ifPDFTeX
  \usepackage[T1]{fontenc}
  \usepackage[utf8]{inputenc}
  \usepackage{textcomp} % provide euro and other symbols
\else % if luatex or xetex
  \usepackage{unicode-math} % this also loads fontspec
  \defaultfontfeatures{Scale=MatchLowercase}
  \defaultfontfeatures[\rmfamily]{Ligatures=TeX,Scale=1}
\fi
\usepackage{lmodern}
\ifPDFTeX\else
  % xetex/luatex font selection
\fi
% Use upquote if available, for straight quotes in verbatim environments
\IfFileExists{upquote.sty}{\usepackage{upquote}}{}
\IfFileExists{microtype.sty}{% use microtype if available
  \usepackage[]{microtype}
  \UseMicrotypeSet[protrusion]{basicmath} % disable protrusion for tt fonts
}{}
\makeatletter
\@ifundefined{KOMAClassName}{% if non-KOMA class
  \IfFileExists{parskip.sty}{%
    \usepackage{parskip}
  }{% else
    \setlength{\parindent}{0pt}
    \setlength{\parskip}{6pt plus 2pt minus 1pt}}
}{% if KOMA class
  \KOMAoptions{parskip=half}}
\makeatother
\usepackage{color}
\usepackage{fancyvrb}
\newcommand{\VerbBar}{|}
\newcommand{\VERB}{\Verb[commandchars=\\\{\}]}
\DefineVerbatimEnvironment{Highlighting}{Verbatim}{commandchars=\\\{\}}
% Add ',fontsize=\small' for more characters per line
\usepackage{framed}
\definecolor{shadecolor}{RGB}{248,248,248}
\newenvironment{Shaded}{\begin{snugshade}}{\end{snugshade}}
\newcommand{\AlertTok}[1]{\textcolor[rgb]{0.94,0.16,0.16}{#1}}
\newcommand{\AnnotationTok}[1]{\textcolor[rgb]{0.56,0.35,0.01}{\textbf{\textit{#1}}}}
\newcommand{\AttributeTok}[1]{\textcolor[rgb]{0.13,0.29,0.53}{#1}}
\newcommand{\BaseNTok}[1]{\textcolor[rgb]{0.00,0.00,0.81}{#1}}
\newcommand{\BuiltInTok}[1]{#1}
\newcommand{\CharTok}[1]{\textcolor[rgb]{0.31,0.60,0.02}{#1}}
\newcommand{\CommentTok}[1]{\textcolor[rgb]{0.56,0.35,0.01}{\textit{#1}}}
\newcommand{\CommentVarTok}[1]{\textcolor[rgb]{0.56,0.35,0.01}{\textbf{\textit{#1}}}}
\newcommand{\ConstantTok}[1]{\textcolor[rgb]{0.56,0.35,0.01}{#1}}
\newcommand{\ControlFlowTok}[1]{\textcolor[rgb]{0.13,0.29,0.53}{\textbf{#1}}}
\newcommand{\DataTypeTok}[1]{\textcolor[rgb]{0.13,0.29,0.53}{#1}}
\newcommand{\DecValTok}[1]{\textcolor[rgb]{0.00,0.00,0.81}{#1}}
\newcommand{\DocumentationTok}[1]{\textcolor[rgb]{0.56,0.35,0.01}{\textbf{\textit{#1}}}}
\newcommand{\ErrorTok}[1]{\textcolor[rgb]{0.64,0.00,0.00}{\textbf{#1}}}
\newcommand{\ExtensionTok}[1]{#1}
\newcommand{\FloatTok}[1]{\textcolor[rgb]{0.00,0.00,0.81}{#1}}
\newcommand{\FunctionTok}[1]{\textcolor[rgb]{0.13,0.29,0.53}{\textbf{#1}}}
\newcommand{\ImportTok}[1]{#1}
\newcommand{\InformationTok}[1]{\textcolor[rgb]{0.56,0.35,0.01}{\textbf{\textit{#1}}}}
\newcommand{\KeywordTok}[1]{\textcolor[rgb]{0.13,0.29,0.53}{\textbf{#1}}}
\newcommand{\NormalTok}[1]{#1}
\newcommand{\OperatorTok}[1]{\textcolor[rgb]{0.81,0.36,0.00}{\textbf{#1}}}
\newcommand{\OtherTok}[1]{\textcolor[rgb]{0.56,0.35,0.01}{#1}}
\newcommand{\PreprocessorTok}[1]{\textcolor[rgb]{0.56,0.35,0.01}{\textit{#1}}}
\newcommand{\RegionMarkerTok}[1]{#1}
\newcommand{\SpecialCharTok}[1]{\textcolor[rgb]{0.81,0.36,0.00}{\textbf{#1}}}
\newcommand{\SpecialStringTok}[1]{\textcolor[rgb]{0.31,0.60,0.02}{#1}}
\newcommand{\StringTok}[1]{\textcolor[rgb]{0.31,0.60,0.02}{#1}}
\newcommand{\VariableTok}[1]{\textcolor[rgb]{0.00,0.00,0.00}{#1}}
\newcommand{\VerbatimStringTok}[1]{\textcolor[rgb]{0.31,0.60,0.02}{#1}}
\newcommand{\WarningTok}[1]{\textcolor[rgb]{0.56,0.35,0.01}{\textbf{\textit{#1}}}}
\usepackage{graphicx}
\makeatletter
\newsavebox\pandoc@box
\newcommand*\pandocbounded[1]{% scales image to fit in text height/width
  \sbox\pandoc@box{#1}%
  \Gscale@div\@tempa{\textheight}{\dimexpr\ht\pandoc@box+\dp\pandoc@box\relax}%
  \Gscale@div\@tempb{\linewidth}{\wd\pandoc@box}%
  \ifdim\@tempb\p@<\@tempa\p@\let\@tempa\@tempb\fi% select the smaller of both
  \ifdim\@tempa\p@<\p@\scalebox{\@tempa}{\usebox\pandoc@box}%
  \else\usebox{\pandoc@box}%
  \fi%
}
% Set default figure placement to htbp
\def\fps@figure{htbp}
\makeatother
\setlength{\emergencystretch}{3em} % prevent overfull lines
\providecommand{\tightlist}{%
  \setlength{\itemsep}{0pt}\setlength{\parskip}{0pt}}
\usepackage{bookmark}
\IfFileExists{xurl.sty}{\usepackage{xurl}}{} % add URL line breaks if available
\urlstyle{same}
\hypersetup{
  pdftitle={Econ 448 Problem Set 4},
  pdfauthor={Alex Lin},
  hidelinks,
  pdfcreator={LaTeX via pandoc}}

\title{Econ 448 Problem Set 4}
\usepackage{etoolbox}
\makeatletter
\providecommand{\subtitle}[1]{% add subtitle to \maketitle
  \apptocmd{\@title}{\par {\large #1 \par}}{}{}
}
\makeatother
\subtitle{Check Canvas for Due Date}
\author{Alex Lin}
\date{12/02/2025}

\begin{document}
\maketitle

\subsection{Concept Questions}\label{concept-questions}

\begin{enumerate}
\def\labelenumi{\arabic{enumi}.}
\tightlist
\item
  In a bargaining model of household decision-making, the couple jointly
  maximizes the following utility function:
  \(U=[U(C_m)- T_m]*[U(C_f)- T_f]\)
\end{enumerate}

Subject to \(Y= C_m+C_f\), where Y is total family income

Person f is the female partner and m is the male partner

1.1 In Nancy Qian's paper, which variable in the preceding model changes
as a direct consequence of an increase in tea prices in China?

Answer:

An increase in tea prices raises women's potential earnings and their
outside options.

The variable that directly varies in this bargaining setup is \(T_f\),
the female partner's threat point (or bargaining power).

1.2 What other model variables will be affected following this change?

Answer:

When \(T_f\) increases, the Nash product
\([U(C_m) - T_m] \times [U(C_f) - T_f]\) is maximized subject to the
same budget \(Y = C_m + C_f\), but with a higher \(T_f\).

To restore optimality:

\begin{itemize}
\tightlist
\item
  The consumption allocation shifts: \(C_f\) increases and \(C_m\)
  decreases, holding \(Y\) fixed.
\item
  As a result, \(U(C_f)\) rises and \(U(C_m)\) may fall.
\end{itemize}

So the affected variables in the simple model are:

\begin{itemize}
\tightlist
\item
  \(C_f\)
\item
  \(C_m\)
\item
  \(U(C_f)\)
\item
  \(U(C_m)\)
\end{itemize}

while \(Y\) and \(T_m\) are taken as given.

1.3 What do Qian's findings regarding girls' survival suggest about the
female partner's preferences in this model?

Answer:

Qian finds that when women's bargaining power rises (via higher tea
prices and thus higher \(T_f\)), the survival rate of girls improves,
while boys are not harmed.

In the context of this model, that implies:

\begin{itemize}
\tightlist
\item
  The female partner's utility \(U(C_f)\) gives relatively high value to
  daughters' survival, health, and well-being.
\item
  When her \(T_f\) increases and she gets a larger share of household
  resources (higher \(C_f\)), she chooses to allocate more resources
  toward girls.
\end{itemize}

This suggests that the female partner's preferences put more weight on
daughters than the male partner's preferences do.

1.4 How does Qian justify rejecting the unitary model of household
decision-making in this paper?

Answer:

Only total income \(Y\) and prices matter in a unitary model since the
household acts as though it maximizes a single utility function; the
distribution of resources within the household shouldn't be influenced
by who makes the money.

Qian shows:

\begin{itemize}
\tightlist
\item
  In areas where tea production is female-labor-intensive, higher tea
  prices (which raise women's income and \(T_f\)) lead to better
  outcomes for girls (higher survival, better investments).
\item
  Girls do not benefit in the same ways from comparable increases in
  income that primarily benefit men (such as from male-intensive
  industries or crops).
\end{itemize}

Because intrahousehold outcomes change when the identity of the income
earner changes, not just when total \(Y\) changes, the unitary model's
key prediction is violated.

\subsection{Data Analysis Questions}\label{data-analysis-questions}

These questions refer to the data set ``1\% sample Ethiopia ever-married
women age 15+'' available via the Econ 448 class account from ipums.org.
Please download the data and answer the following questions.

\begin{Shaded}
\begin{Highlighting}[]
\FunctionTok{library}\NormalTok{(tidyverse)}
\end{Highlighting}
\end{Shaded}

\begin{verbatim}
## -- Attaching core tidyverse packages ------------------------ tidyverse 2.0.0 --
## v dplyr     1.1.4     v readr     2.1.5
## v forcats   1.0.0     v stringr   1.5.1
## v ggplot2   3.5.2     v tibble    3.2.1
## v lubridate 1.9.4     v tidyr     1.3.1
## v purrr     1.0.4     
## -- Conflicts ------------------------------------------ tidyverse_conflicts() --
## x dplyr::filter() masks stats::filter()
## x dplyr::lag()    masks stats::lag()
## i Use the conflicted package (<http://conflicted.r-lib.org/>) to force all conflicts to become errors
\end{verbatim}

\begin{Shaded}
\begin{Highlighting}[]
\NormalTok{data\_path }\OtherTok{\textless{}{-}} \StringTok{"C:/Users/alexl/OneDrive/Desktop/桌面/UW/CourseWork/Senior Year/Autumn 2025/ECON 448/Problem Set 4/ipumsi\_00003.csv.gz"}

\NormalTok{raw }\OtherTok{\textless{}{-}}\NormalTok{ readr}\SpecialCharTok{::}\FunctionTok{read\_csv}\NormalTok{(data\_path)}
\end{Highlighting}
\end{Shaded}

\begin{verbatim}
## Rows: 28759 Columns: 36
## -- Column specification --------------------------------------------------------
## Delimiter: ","
## dbl (36): COUNTRY, YEAR, SAMPLE, SERIAL, HHWT, FORMTYPE, URBAN, OWNERSHIP, O...
## 
## i Use `spec()` to retrieve the full column specification for this data.
## i Specify the column types or set `show_col_types = FALSE` to quiet this message.
\end{verbatim}

\begin{enumerate}
\def\labelenumi{\arabic{enumi}.}
\tightlist
\item
  Clean your data, as shown in the IPUMS and Child Mortality Labs.
  Create a formatted table of summary statistics (mean, std dev, max,
  min) for all variables that you use in your analysis (Hint: wait until
  you answer all of the other questions before finalizing this part).
  Write one paragraph describing your data using the numbers in this
  table.
\end{enumerate}

\begin{Shaded}
\begin{Highlighting}[]
\NormalTok{df }\OtherTok{\textless{}{-}}\NormalTok{ raw }\SpecialCharTok{\%\textgreater{}\%}
  \FunctionTok{filter}\NormalTok{(}
\NormalTok{    AGE }\SpecialCharTok{\textgreater{}=} \DecValTok{15}\NormalTok{,}
\NormalTok{    SEX }\SpecialCharTok{==} \DecValTok{2}
\NormalTok{  ) }\SpecialCharTok{\%\textgreater{}\%}
  \FunctionTok{mutate}\NormalTok{(}
    \AttributeTok{CHSURV\_CLEAN =} \FunctionTok{if\_else}\NormalTok{(CHSURV }\SpecialCharTok{\textgreater{}=} \DecValTok{90}\NormalTok{, }\ConstantTok{NA\_real\_}\NormalTok{, CHSURV),}

    \AttributeTok{CHILDREN\_EVER\_BORN =}\NormalTok{ CHBORN,}
    \AttributeTok{CHILDREN\_SURVIVING =}\NormalTok{ CHSURV\_CLEAN,}
    \AttributeTok{CHILDREN\_DEAD =} \FunctionTok{if\_else}\NormalTok{(}
      \SpecialCharTok{!}\FunctionTok{is.na}\NormalTok{(CHILDREN\_EVER\_BORN) }\SpecialCharTok{\&} \SpecialCharTok{!}\FunctionTok{is.na}\NormalTok{(CHILDREN\_SURVIVING),}
\NormalTok{      CHILDREN\_EVER\_BORN }\SpecialCharTok{{-}}\NormalTok{ CHILDREN\_SURVIVING,}
      \ConstantTok{NA\_real\_}
\NormalTok{    ),}

    \AttributeTok{CM\_RATE =} \FunctionTok{if\_else}\NormalTok{(}
\NormalTok{      CHILDREN\_EVER\_BORN }\SpecialCharTok{\textgreater{}} \DecValTok{0} \SpecialCharTok{\&} \SpecialCharTok{!}\FunctionTok{is.na}\NormalTok{(CHILDREN\_DEAD),}
\NormalTok{      CHILDREN\_DEAD }\SpecialCharTok{/}\NormalTok{ CHILDREN\_EVER\_BORN,}
      \ConstantTok{NA\_real\_}
\NormalTok{    ),}

    \AttributeTok{EDUC\_W =} \FunctionTok{if\_else}\NormalTok{(YRSCHOOL }\SpecialCharTok{\textgreater{}=} \DecValTok{90}\NormalTok{, }\ConstantTok{NA\_real\_}\NormalTok{, YRSCHOOL),}
    \AttributeTok{EDUC\_SP =} \FunctionTok{if\_else}\NormalTok{(YRSCHOOL\_SP }\SpecialCharTok{\textgreater{}=} \DecValTok{90}\NormalTok{, }\ConstantTok{NA\_real\_}\NormalTok{, YRSCHOOL\_SP),}
    \AttributeTok{URBAN\_D =} \FunctionTok{if\_else}\NormalTok{(URBAN }\SpecialCharTok{==} \DecValTok{1}\NormalTok{, }\DecValTok{1}\NormalTok{, }\DecValTok{0}\NormalTok{),}

    \AttributeTok{ELECTRIC\_D =} \FunctionTok{case\_when}\NormalTok{(}
\NormalTok{      ELECTRIC }\SpecialCharTok{==} \DecValTok{1} \SpecialCharTok{\textasciitilde{}} \DecValTok{1}\NormalTok{,}
\NormalTok{      ELECTRIC }\SpecialCharTok{==} \DecValTok{2} \SpecialCharTok{\textasciitilde{}} \DecValTok{0}\NormalTok{,}
      \ConstantTok{TRUE}          \SpecialCharTok{\textasciitilde{}} \ConstantTok{NA\_real\_}
\NormalTok{    )}
\NormalTok{  ) }\SpecialCharTok{\%\textgreater{}\%}
  \FunctionTok{filter}\NormalTok{(}
    \SpecialCharTok{!}\FunctionTok{is.na}\NormalTok{(CM\_RATE),}
    \SpecialCharTok{!}\FunctionTok{is.na}\NormalTok{(EDUC\_W),}
    \SpecialCharTok{!}\FunctionTok{is.na}\NormalTok{(AGE),}
    \SpecialCharTok{!}\FunctionTok{is.na}\NormalTok{(URBAN\_D),}
    \SpecialCharTok{!}\FunctionTok{is.na}\NormalTok{(ELECTRIC\_D),}
    \SpecialCharTok{!}\FunctionTok{is.na}\NormalTok{(PERWT)}
\NormalTok{  )}

\NormalTok{vars\_for\_analysis }\OtherTok{\textless{}{-}}\NormalTok{ df }\SpecialCharTok{\%\textgreater{}\%}
  \FunctionTok{select}\NormalTok{(}
\NormalTok{    CM\_RATE,}
\NormalTok{    EDUC\_W,}
\NormalTok{    EDUC\_SP,}
\NormalTok{    AGE,}
\NormalTok{    URBAN\_D,}
\NormalTok{    ELECTRIC\_D,}
\NormalTok{    CHBORN,}
\NormalTok{    CHSURV\_CLEAN,}
\NormalTok{    PERWT}
\NormalTok{  )}
\end{Highlighting}
\end{Shaded}

\begin{Shaded}
\begin{Highlighting}[]
\NormalTok{summary\_table\_long }\OtherTok{\textless{}{-}}\NormalTok{ vars\_for\_analysis }\SpecialCharTok{\%\textgreater{}\%}
  \FunctionTok{select}\NormalTok{(}\SpecialCharTok{{-}}\NormalTok{PERWT) }\SpecialCharTok{\%\textgreater{}\%}    \CommentTok{\# simple unweighted summary}
  \FunctionTok{pivot\_longer}\NormalTok{(}\AttributeTok{cols =} \FunctionTok{everything}\NormalTok{()) }\SpecialCharTok{\%\textgreater{}\%}
  \FunctionTok{group\_by}\NormalTok{(name) }\SpecialCharTok{\%\textgreater{}\%}
  \FunctionTok{summarise}\NormalTok{(}
    \AttributeTok{mean =} \FunctionTok{mean}\NormalTok{(value, }\AttributeTok{na.rm =} \ConstantTok{TRUE}\NormalTok{),}
    \AttributeTok{sd   =} \FunctionTok{sd}\NormalTok{(value,   }\AttributeTok{na.rm =} \ConstantTok{TRUE}\NormalTok{),}
    \AttributeTok{min  =} \FunctionTok{min}\NormalTok{(value,  }\AttributeTok{na.rm =} \ConstantTok{TRUE}\NormalTok{),}
    \AttributeTok{max  =} \FunctionTok{max}\NormalTok{(value,  }\AttributeTok{na.rm =} \ConstantTok{TRUE}\NormalTok{),}
    \AttributeTok{.groups =} \StringTok{"drop"}
\NormalTok{  )}

\NormalTok{summary\_table\_long}
\end{Highlighting}
\end{Shaded}

\begin{verbatim}
## # A tibble: 8 x 5
##   name           mean     sd   min    max
##   <chr>         <dbl>  <dbl> <dbl>  <dbl>
## 1 AGE          36.9   14.5      15 97    
## 2 CHBORN        4.44   2.51      1  9    
## 3 CHSURV_CLEAN  3.81   2.14      1  9    
## 4 CM_RATE       0.103  0.180     0  0.889
## 5 EDUC_SP       1.84   3.25      0 13    
## 6 EDUC_W        0.859  2.42      0 13    
## 7 ELECTRIC_D    0.132  0.338     0  1    
## 8 URBAN_D       0.856  0.351     0  1
\end{verbatim}

Ethiopian women who were married and at least 15 years old in 2007 make
up the analytical sample. The average age of a woman is approximately
36.9 (sd ≈ 14.5, range 15--97). The average number of children born to
women in the sample is 4.44 (sd ≈ 2.51), with values ranging from 1 to
9. Among women with at least one birth, the mean number of surviving
children is 3.81 (sd ≈ 2.14, range 1--9). The constructed child
mortality rate ranges from 0 (no child deaths) to 0.889 (about 8 deaths
out of 9 births), with a mean of roughly 0.103 (sd ≈ 0.18). It is
calculated by dividing the number of children who have died by the
number of children who have been born.

This sample has extremely low levels of education. Husbands have an
average of 1.84 years (sd ≈ 3.25, max 13) of education, while women have
an average of 0.86 years (sd ≈ 2.42, max 13). Only about 13 percent of
households have access to electricity (mean of the electricity dummy =
0.132), which is consistent with a generally poor, low-infrastructure
setting. Based on the urban dummy used here, roughly 86 percent of women
are classified as living in urban areas (mean of URBAN\_D = 0.856).
Overall, the summary statistics describe a high-fertility, low-education
population with substantial variation in child mortality and household
infrastructure.

\begin{enumerate}
\def\labelenumi{\arabic{enumi}.}
\setcounter{enumi}{1}
\tightlist
\item
  Regress child mortality rate on women's education, include control
  variables for age, urban residence (create a dummy variable for
  residence in an urban area), and at least one other control variable
  of your choosing. Tell me why you included that control variable and
  what you expect the sign to be for the the coefficients on that
  control variable.
\end{enumerate}

\begin{Shaded}
\begin{Highlighting}[]
\NormalTok{ols\_mod }\OtherTok{\textless{}{-}} \FunctionTok{lm}\NormalTok{(}
\NormalTok{  CM\_RATE }\SpecialCharTok{\textasciitilde{}}\NormalTok{ EDUC\_W }\SpecialCharTok{+}\NormalTok{ AGE }\SpecialCharTok{+}\NormalTok{ URBAN\_D }\SpecialCharTok{+}\NormalTok{ ELECTRIC\_D,}
  \AttributeTok{data    =}\NormalTok{ df,}
  \AttributeTok{weights =}\NormalTok{ PERWT}
\NormalTok{)}

\FunctionTok{summary}\NormalTok{(ols\_mod)}
\end{Highlighting}
\end{Shaded}

\begin{verbatim}
## 
## Call:
## lm(formula = CM_RATE ~ EDUC_W + AGE + URBAN_D + ELECTRIC_D, data = df, 
##     weights = PERWT)
## 
## Weighted Residuals:
##    Min     1Q Median     3Q    Max 
## -8.648 -2.323 -1.352  1.143 20.936 
## 
## Coefficients:
##               Estimate Std. Error t value Pr(>|t|)    
## (Intercept) -3.206e-02  6.210e-03  -5.163 2.45e-07 ***
## EDUC_W      -4.655e-03  5.494e-04  -8.472  < 2e-16 ***
## AGE          3.611e-03  8.084e-05  44.677  < 2e-16 ***
## URBAN_D      7.006e-03  5.347e-03   1.310    0.190    
## ELECTRIC_D  -4.454e-03  5.596e-03  -0.796    0.426    
## ---
## Signif. codes:  0 '***' 0.001 '**' 0.01 '*' 0.05 '.' 0.1 ' ' 1
## 
## Residual standard error: 4.002 on 22381 degrees of freedom
## Multiple R-squared:  0.09403,    Adjusted R-squared:  0.09387 
## F-statistic: 580.7 on 4 and 22381 DF,  p-value: < 2.2e-16
\end{verbatim}

I estimate a weighted OLS regression of the child mortality rate on
women's education and several controls:

\[
CM\_RATE_i = \beta_0 + \beta_1 EDUC\_W_i + \beta_2 AGE_i + \beta_3 URBAN\_D_i + \beta_4 ELECTRIC\_D_i + \varepsilon_i$, using $PERWT
\]

as person weights. The key explanatory variable is women's years of
schooling (\(EDUC\_W\)).

I control for \(AGE\) because older women have had more time to
accumulate births and potential child deaths, so even holding education
constant, we expect higher exposure to mortality events at older ages.
As a result, I expected \(AGE\) to have a positive coefficient. Because
urban areas usually have better access to clinics, hospitals, clean
water, and other health-related infrastructure, I included an urban
residence dummy (\(URBAN\_D = 1\) if urban). I expected this to be
associated with lower child mortality, so a negative coefficient on
\(URBAN\_D\). Lastly, as a stand-in for household wealth and
infrastructure, I include an electricity dummy (\(ELECTRIC\_D = 1\) if
the household has electricity). I anticipated that \(ELECTRIC_D\) would
also have a negative coefficient since households with electricity are
probably wealthier and better able to invest in children's health.

\begin{enumerate}
\def\labelenumi{\arabic{enumi}.}
\setcounter{enumi}{2}
\tightlist
\item
  Display your regression results from 2 in a table. Label your
  variables in the table and give your table a title. Were the signs on
  the coefficients as expected?
\end{enumerate}

\begin{Shaded}
\begin{Highlighting}[]
\NormalTok{ols\_sum }\OtherTok{\textless{}{-}} \FunctionTok{summary}\NormalTok{(ols\_mod)}

\NormalTok{ols\_table }\OtherTok{\textless{}{-}} \FunctionTok{as.data.frame}\NormalTok{(ols\_sum}\SpecialCharTok{$}\NormalTok{coefficients)}
\NormalTok{ols\_table}\SpecialCharTok{$}\NormalTok{Variable }\OtherTok{\textless{}{-}} \FunctionTok{rownames}\NormalTok{(ols\_table)}
\FunctionTok{rownames}\NormalTok{(ols\_table) }\OtherTok{\textless{}{-}} \ConstantTok{NULL}
\FunctionTok{colnames}\NormalTok{(ols\_table)[}\DecValTok{1}\SpecialCharTok{:}\DecValTok{4}\NormalTok{] }\OtherTok{\textless{}{-}} \FunctionTok{c}\NormalTok{(}\StringTok{"Estimate"}\NormalTok{, }\StringTok{"Std\_Error"}\NormalTok{, }\StringTok{"t\_value"}\NormalTok{, }\StringTok{"Pr\_t"}\NormalTok{)}

\NormalTok{ols\_table\_pretty }\OtherTok{\textless{}{-}}\NormalTok{ ols\_table }\SpecialCharTok{\%\textgreater{}\%}
  \FunctionTok{mutate}\NormalTok{(}
    \AttributeTok{Variable =} \FunctionTok{case\_when}\NormalTok{(}
\NormalTok{      Variable }\SpecialCharTok{==} \StringTok{"(Intercept)"} \SpecialCharTok{\textasciitilde{}} \StringTok{"Intercept"}\NormalTok{,}
\NormalTok{      Variable }\SpecialCharTok{==} \StringTok{"EDUC\_W"}      \SpecialCharTok{\textasciitilde{}} \StringTok{"Years of schooling (woman)"}\NormalTok{,}
\NormalTok{      Variable }\SpecialCharTok{==} \StringTok{"AGE"}         \SpecialCharTok{\textasciitilde{}} \StringTok{"Age"}\NormalTok{,}
\NormalTok{      Variable }\SpecialCharTok{==} \StringTok{"URBAN\_D"}     \SpecialCharTok{\textasciitilde{}} \StringTok{"Urban (=1)"}\NormalTok{,}
\NormalTok{      Variable }\SpecialCharTok{==} \StringTok{"ELECTRIC\_D"}  \SpecialCharTok{\textasciitilde{}} \StringTok{"Electricity (=1)"}\NormalTok{,}
      \ConstantTok{TRUE}                      \SpecialCharTok{\textasciitilde{}}\NormalTok{ Variable}
\NormalTok{    )}
\NormalTok{  ) }\SpecialCharTok{\%\textgreater{}\%}
  \FunctionTok{select}\NormalTok{(Variable, Estimate, Std\_Error, t\_value, Pr\_t)}

\NormalTok{ols\_table\_pretty}
\end{Highlighting}
\end{Shaded}

\begin{verbatim}
##                     Variable     Estimate    Std_Error    t_value         Pr_t
## 1                  Intercept -0.032060240 6.209795e-03 -5.1628499 2.452984e-07
## 2 Years of schooling (woman) -0.004654806 5.494200e-04 -8.4722186 2.554000e-17
## 3                        Age  0.003611487 8.083575e-05 44.6768529 0.000000e+00
## 4                 Urban (=1)  0.007005714 5.347147e-03  1.3101779 1.901491e-01
## 5           Electricity (=1) -0.004454202 5.596127e-03 -0.7959438 4.260731e-01
\end{verbatim}

Table 1 reports the weighted OLS regression of the child mortality rate
on women's years of schooling, controlling for age, urban residence, and
electricity access. The coefficient on women's education is −0.00465 and
highly statistically significant. This suggests that an extra year of
education is linked to a roughly 0.47 percentage point decrease in the
child mortality rate, all other things being equal. The expectation that
mothers with higher levels of education have lower rates of child
mortality is supported by the negative sign.

The idea that older women have had more time to experience child deaths
is supported by the coefficient on age, which is 0.00361 and highly
significant. This means that every extra year of age is linked to a 0.36
percentage point higher child mortality rate. In contrast to my earlier
prediction of lower mortality in urban areas, the coefficient on urban
residence is positive (0.0070) and not statistically significant at
conventional levels. The electricity dummy is statistically
insignificant despite having the anticipated negative sign (-0.00445).
While the coefficients on urban residence and electricity are weaker and
less accurately estimated than expected, the signs for women's age and
education are generally as expected.

\begin{enumerate}
\def\labelenumi{\arabic{enumi}.}
\setcounter{enumi}{3}
\tightlist
\item
  Why might the estimated relationship between child mortality and
  women's education you estimated above be biased? Even though this is
  the data science section, draw on at least one of the course readings
  to support your argument.
\end{enumerate}

Because education is not assigned at random, the negative OLS
association between women's education and child mortality is unlikely to
be solely causal. Women who obtain more schooling tend to come from
households with higher income, better nutrition, and stronger
preferences for child quality, and they may live in communities with
better health services. These invisible traits have an impact on
children's survival as well as their level of education. The OLS
coefficient on women's education will capture both the actual causal
effect of education and the influence of these advantageous background
factors, resulting in bias, if we fail to fully observe or account for
them.

The concerns brought up in the course readings on health and education
are strikingly similar to this one. For example, in our reading on
maternal education and child health (where the authors emphasize that
more educated mothers also come from households with better health
environments and higher permanent income), the authors argue that simple
cross-sectional regressions overstate the causal impact of education
because they confound schooling with family background and community
characteristics. Similarly, our OLS results for Ethiopia likely
overestimate the amount that child mortality would decrease if we
exogenously increased women's schooling because favorable family and
community circumstances that are correlated with education rather than
education itself account for a portion of the observed gap in child
mortality.

\begin{enumerate}
\def\labelenumi{\arabic{enumi}.}
\setcounter{enumi}{4}
\tightlist
\item
  Using her spouse's education as an instrumental variable for a woman's
  education, perform two stage least squares. Display the results of
  your second stage in a table that meets the same criteria as the table
  you created in 3.
\end{enumerate}

\begin{Shaded}
\begin{Highlighting}[]
\FunctionTok{library}\NormalTok{(AER)}
\end{Highlighting}
\end{Shaded}

\begin{verbatim}
## Warning: 套件 'AER' 是用 R 版本 4.5.2 來建造的
\end{verbatim}

\begin{verbatim}
## 載入需要的套件:car
\end{verbatim}

\begin{verbatim}
## Warning: 套件 'car' 是用 R 版本 4.5.2 來建造的
\end{verbatim}

\begin{verbatim}
## 載入需要的套件:carData
\end{verbatim}

\begin{verbatim}
## Warning: 套件 'carData' 是用 R 版本 4.5.2 來建造的
\end{verbatim}

\begin{verbatim}
## 
## 載入套件:'car'
\end{verbatim}

\begin{verbatim}
## 下列物件被遮斷自 'package:dplyr':
## 
##     recode
\end{verbatim}

\begin{verbatim}
## 下列物件被遮斷自 'package:purrr':
## 
##     some
\end{verbatim}

\begin{verbatim}
## 載入需要的套件:lmtest
\end{verbatim}

\begin{verbatim}
## Warning: 套件 'lmtest' 是用 R 版本 4.5.2 來建造的
\end{verbatim}

\begin{verbatim}
## 載入需要的套件:zoo
\end{verbatim}

\begin{verbatim}
## 
## 載入套件:'zoo'
\end{verbatim}

\begin{verbatim}
## 下列物件被遮斷自 'package:base':
## 
##     as.Date, as.Date.numeric
\end{verbatim}

\begin{verbatim}
## 載入需要的套件:sandwich
\end{verbatim}

\begin{verbatim}
## Warning: 套件 'sandwich' 是用 R 版本 4.5.2 來建造的
\end{verbatim}

\begin{verbatim}
## 載入需要的套件:survival
\end{verbatim}

\begin{Shaded}
\begin{Highlighting}[]
\NormalTok{iv\_mod }\OtherTok{\textless{}{-}} \FunctionTok{ivreg}\NormalTok{(}
\NormalTok{  CM\_RATE }\SpecialCharTok{\textasciitilde{}}\NormalTok{ EDUC\_W }\SpecialCharTok{+}\NormalTok{ AGE }\SpecialCharTok{+}\NormalTok{ URBAN\_D }\SpecialCharTok{+}\NormalTok{ ELECTRIC\_D }\SpecialCharTok{|}
\NormalTok{    EDUC\_SP }\SpecialCharTok{+}\NormalTok{ AGE }\SpecialCharTok{+}\NormalTok{ URBAN\_D }\SpecialCharTok{+}\NormalTok{ ELECTRIC\_D,}
  \AttributeTok{data    =}\NormalTok{ df,}
  \AttributeTok{weights =}\NormalTok{ PERWT}
\NormalTok{)}

\FunctionTok{summary}\NormalTok{(iv\_mod, }\AttributeTok{diagnostics =} \ConstantTok{TRUE}\NormalTok{)}
\end{Highlighting}
\end{Shaded}

\begin{verbatim}
## 
## Call:
## ivreg(formula = CM_RATE ~ EDUC_W + AGE + URBAN_D + ELECTRIC_D | 
##     EDUC_SP + AGE + URBAN_D + ELECTRIC_D, data = df, weights = PERWT)
## 
## Residuals:
##    Min     1Q Median     3Q    Max 
## -7.055 -1.972 -1.319  0.233 20.879 
## 
## Coefficients:
##               Estimate Std. Error t value Pr(>|t|)    
## (Intercept) -0.0244634  0.0078553  -3.114  0.00185 ** 
## EDUC_W      -0.0038083  0.0011798  -3.228  0.00125 ** 
## AGE          0.0031056  0.0001085  28.617  < 2e-16 ***
## URBAN_D      0.0104130  0.0064935   1.604  0.10882    
## ELECTRIC_D  -0.0063899  0.0069302  -0.922  0.35652    
## 
## Diagnostic tests:
##                    df1   df2 statistic p-value    
## Weak instruments     1 15607  5822.858  <2e-16 ***
## Wu-Hausman           1 15606     0.022   0.882    
## Sargan               0    NA        NA      NA    
## ---
## Signif. codes:  0 '***' 0.001 '**' 0.01 '*' 0.05 '.' 0.1 ' ' 1
## 
## Residual standard error: 3.678 on 15607 degrees of freedom
## Multiple R-Squared: 0.06158, Adjusted R-squared: 0.06044 
## Wald test: 248.3 on 4 and 15607 DF,  p-value: < 2.2e-16
\end{verbatim}

I estimate the following second--stage equation using two--stage least
squares (2SLS):

\[
CM\_RATE_i = \beta_0 + \beta_1 EDUC\_W_i + \beta_2 AGE_i + \beta_3 URBAN\_D_i + \beta_4 ELECTRIC\_D_i + \varepsilon_i
\] where \(EDUC_W\) (woman's years of schooling) is instrumented by
\(EDUC_SP\) (spouse's years of schooling), and the controls \(AGE\),
\(URBAN_D\), and \(ELECTRIC_D\) enter both stages. According to the
weighted 2SLS results, women's education has a coefficient of
\(-0.00381\) (standard error \(0.00118\), \(p = 0.0013\)). This implies
that, for the women whose schooling is shifted by their spouse's
education, an additional year of schooling is associated with roughly a
\(0.38\) percentage point reduction in the child mortality rate, holding
age, urban residence, and electricity access constant. The coefficient
on \(AGE\) remains positive and highly significant (\(0.00311\)), while
the coefficients on \(URBAN_D\) and \(ELECTRIC_D\) are small and
statistically insignificant, similar to the OLS results.

The diagnostic tests indicate that \(EDUC_SP\) is an extremely strong
instrument: the weak‐instrument \(F\)--statistic is about \(5823\) with
\(p < 0.001\), far above conventional thresholds. With a \(p\)--value of
\(0.882\) for the Wu--Hausman test, we are unable to reject the null
hypothesis that OLS is consistent; in other words, there is no
compelling statistical evidence that endogeneity of \(EDUC_W\) is
causing a significant discrepancy between the OLS and 2SLS estimates.
Consistent with this, the 2SLS estimate (\(-0.0038\)) is somewhat
smaller in magnitude than the OLS estimate (\(-0.0047\)) but still
negative and statistically significant.

\begin{enumerate}
\def\labelenumi{\arabic{enumi}.}
\setcounter{enumi}{5}
\tightlist
\item
  Write one paragraph discussing how the results you found in your OLS
  and two stage least squares regressions (\#2 and \#5) relate to the
  theoretical model(s) learned in the course. Think of this as a
  practice run for your paper.
\end{enumerate}

According to the theoretical models we covered in class, raising women's
educational attainment should lower child mortality by enhancing
maternal health and nutrition knowledge, changing preferences for
high-quality children, and possibly bolstering women's negotiating power
in the home. This prediction is supported by both my OLS and 2SLS
estimates; in each instance, the coefficient on \(EDUC_W\) is negative
and statistically significant, suggesting that lower child mortality is
linked to higher maternal education. The goal of the 2SLS method, which
employs \(EDUC_SP\) as a tool, is to separate the portion of women's
education that is influenced by assortative matching in the marriage
market as opposed to unreported family history. The fact that the 2SLS
estimate is still negative and of comparable order of magnitude to the
OLS estimate indicates that the observed relationship is not solely due
to selection on unobservables and that further education improves child
survival causally, at least for the ``compliers'' whose education
responds to their spouse's education. However, the Wu-Hausman test and
the slight difference between OLS and 2SLS indicate relatively mild bias
in the simple regression. Therefore, the overall empirical evidence
supports the theoretical view that maternal education is a significant,
though not exclusive, factor in lowering child mortality in this
context.

\end{document}
